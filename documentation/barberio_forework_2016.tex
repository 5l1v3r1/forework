%%%%%%%%%%%%%%%%%%%%%%%%%%%%%%%%%%%%%%%%%
% Journal Article
% LaTeX Template
% Version 1.3 (9/9/13)
%
% This template has been downloaded from:
% http://www.LaTeXTemplates.com
%
% Original author:
% Frits Wenneker (http://www.howtotex.com)
%
% License:
% CC BY-NC-SA 3.0 (http://creativecommons.org/licenses/by-nc-sa/3.0/)
%
%%%%%%%%%%%%%%%%%%%%%%%%%%%%%%%%%%%%%%%%%

%----------------------------------------------------------------------------------------
%	PACKAGES AND OTHER DOCUMENT CONFIGURATIONS
%----------------------------------------------------------------------------------------

\documentclass[twoside]{article}

\usepackage{lipsum} % Package to generate dummy text throughout this template

\usepackage[sc]{mathpazo} % Use the Palatino font
\usepackage[T1]{fontenc} % Use 8-bit encoding that has 256 glyphs
\linespread{1.05} % Line spacing - Palatino needs more space between lines
\usepackage{microtype} % Slightly tweak font spacing for aesthetics

\usepackage[hmarginratio=1:1,top=32mm,columnsep=20pt]{geometry} % Document margins
\usepackage{multicol} % Used for the two-column layout of the document
\usepackage[hang, small,labelfont=bf,up,textfont=it,up]{caption} % Custom captions under/above floats in tables or figures
\usepackage{booktabs} % Horizontal rules in tables
\usepackage{float} % Required for tables and figures in the multi-column environment - they need to be placed in specific locations with the [H] (e.g. \begin{table}[H])
\usepackage{hyperref} % For hyperlinks in the PDF

\usepackage{lettrine} % The lettrine is the first enlarged letter at the beginning of the text
\usepackage{paralist} % Used for the compactitem environment which makes bullet points with less space between them

\usepackage{abstract} % Allows abstract customization
\renewcommand{\abstractnamefont}{\normalfont\bfseries} % Set the "Abstract" text to bold
\renewcommand{\abstracttextfont}{\normalfont\small\itshape} % Set the abstract itself to small italic text

\usepackage{titlesec} % Allows customization of titles
\renewcommand\thesection{\Roman{section}} % Roman numerals for the sections
\renewcommand\thesubsection{\Roman{subsection}} % Roman numerals for subsections
\titleformat{\section}[block]{\large\scshape\centering}{\thesection.}{1em}{} % Change the look of the section titles
\titleformat{\subsection}[block]{\large}{\thesubsection.}{1em}{} % Change the look of the section titles

\usepackage{fancyhdr} % Headers and footers
\pagestyle{fancy} % All pages have headers and footers
\fancyhead{} % Blank out the default header
\fancyfoot{} % Blank out the default footer
%\fancyhead[C]{ForeWork: a distributed framework for automated extraction of triaged forensic artifacts $\bullet$ August 2016} % Custom header text
\fancyfoot[RO,LE]{\thepage} % Custom footer text

%----------------------------------------------------------------------------------------
%	TITLE SECTION
%----------------------------------------------------------------------------------------

\title{\vspace{-15mm}\fontsize{24pt}{10pt}\selectfont\textbf{ForeWork,
    a distributed real-time framework for automated extraction of
    triaged forensic artifacts}} % Article title
\author{\large
\textsc{Andrea Barberio}\thanks{A thank you or further information}\\[2mm] % Your name
\normalsize University College Dublin \\ % Your institution
\normalsize
    \{
        \href{mailto:andrea.barberio@ucdconnect.ie}{andrea.barberio@ucdconnect.ie},
        \href{mailto:insomniac@slackware.it}{insomniac@slackware.it}
    \}
    % Your email address
\vspace{-5mm}
}
\date{}

%----------------------------------------------------------------------------------------

\begin{document}

\maketitle % Insert title

\thispagestyle{fancy} % All pages have headers and footers

%----------------------------------------------------------------------------------------
%	ABSTRACT
%----------------------------------------------------------------------------------------

\begin{abstract}

% abstract topics: question/purpose, design and methods, major findings,
% results, interpretation and conclusions
\noindent
    % intro
    This paper describes ForeWork, a novel software framework for
    simplifying and automating the extraction of forensic artifacts in a
    distributed and efficient manner.
    % question/purpose
    Modern investigations can involve large amounts of data, often requiring
    long time in order to be analyzed.
    Digital evidence shows a bigger impact if it is found within the first few
    days since the start of the trial.
%    This means that an investigation
%    involving a large amount of data may result less effective if the most
%    important evidence is not obtained in time.
    The existing distributed forensic frameworks focus on extracting all the artifacts
    in the shortest time span possible, but in no special order.
    ForeWork instead puts more emphasis on extracting the artifacts that the
    investigator considers more important for the case.
    % design and methods
    To achieve this, we use a parallel, distributed, horizontally
    scalable, stream-processing  approach, that is not found in other forensic
    softwares.
%    % other main goals
%    % This paragraph maybe can be removed
%    \textbf{Other main goals of ForeWork are: interactivity (the
%    forensic examiner can observe any artifact in detail, and conduct in-depth manual
%    analysis), ease of extensibility (no special programming skill is required
%    to implement new analysis plugins), ease of adoption (it is written in an
%    approachable and widespread language, Python 3, that makes adoption and
%    contribution easier) and friendly licensing (it is open source, under the
%    permissive BSD licence, suitable for both the open source and the commercial
%    worlds).
%    }
    % findings and conclusions
    In this paper we show how the distributed, stream processing approach,
    together with prioritizing the extraction of the most important artifacts,
    can help maximize the chances of obtaining useful forensic evidence in
    shorter time, compared to a regular distributed batch-processing framework.
% \noindent \lipsum[1] % Dummy abstract text

\end{abstract}

%----------------------------------------------------------------------------------------
%	ARTICLE CONTENTS
%----------------------------------------------------------------------------------------

\begin{multicols}{2} % Two-column layout throughout the main article text

\section{Introduction}

\lettrine[nindent=0em,lines=3]{L} orem ipsum dolor sit amet, consectetur adipiscing elit.
\lipsum[2-3] % Dummy text


%------------------------------------------------

\section{Methods}

Maecenas sed ultricies felis. Sed imperdiet dictum arcu a egestas. 
\begin{compactitem}
\item Donec dolor arcu, rutrum id molestie in, viverra sed diam
\item Curabitur feugiat
\item turpis sed auctor facilisis
\item arcu eros accumsan lorem, at posuere mi diam sit amet tortor
\item Fusce fermentum, mi sit amet euismod rutrum
\item sem lorem molestie diam, iaculis aliquet sapien tortor non nisi
\item Pellentesque bibendum pretium aliquet
\end{compactitem}
\lipsum[4] % Dummy text

%------------------------------------------------

\section{Results}

\begin{table}[H]
\caption{Example table}
\centering
\begin{tabular}{llr}
\toprule
\multicolumn{2}{c}{Name} \\
\cmidrule(r){1-2}
First name & Last Name & Grade \\
\midrule
John & Doe & $7.5$ \\
Richard & Miles & $2$ \\
\bottomrule
\end{tabular}
\end{table}

\lipsum[5] % Dummy text

\begin{equation}
\label{eq:emc}
e = mc^2
\end{equation}

\lipsum[6] % Dummy text

%------------------------------------------------

\section{Discussion}

\subsection{Subsection One}

\lipsum[7] % Dummy text

\subsection{Subsection Two}

\lipsum[8] % Dummy text

%----------------------------------------------------------------------------------------
%	REFERENCE LIST
%----------------------------------------------------------------------------------------

\begin{thebibliography}{99} % Bibliography - this is intentionally simple in this template

\bibitem{vanbeek2015}
    H.M.A van Beek, E.J. van Eijk, R.B. van Baar, M. Ugen, J.N.C. Bodde, A.J. Siemelink,
    \emph{Digital forensics as a service: Game On},
    Netherlands Forensic Institute,
    December 2015

\end{thebibliography}

%----------------------------------------------------------------------------------------

\end{multicols}

\end{document}
